\chapter{應用情境}

\hspace{2em}當新學期開始,你正準備修習幾門新課,卻發現教材清單長得嚇人,有些書在圖書館早已被借光,有些則不知道該去哪裡找。你想要快速掌握每門課的指定用書、看看有沒有人分享過使用心得,甚至希望能線上預約或直接閱讀電子版。這時,「校園線上書城」就是你最好的幫手——它整合課程書單、圖書館藏、電子資源,讓你在一個平台上就能輕鬆找到或借閱需要的書籍。不再奔波於各大網站或書局,專注學習就是這麼簡單。以下是幾個具體的應用情境:

\begin{itemize}
    \item 線上借書,以數位化的方式線上閱讀書籍
        \begin{enumerate}
            \item 情境描述:張小華是一名大三生,期末報告需要多本參考書。他打開校園線上書城,先以學校帳號快速登入。接著,他在搜尋欄輸入課程名稱,立即看到《現代資料庫系統概論》的電子書和實體書選項。考慮到時間緊迫,他選擇了電子書借閱,系統依照出版社授權自動核准並顯示借閱期限與閱讀須知。確認後,他一邊在網頁上閱讀,還能隨時標註重點;到期前系統也在首頁醒目位置提醒他延長或歸還。整個流程不到三分鐘,他就完成借閱並開始專注閱讀,無需排隊或等待實體還書。
            \item 系統應用:
                \begin{enumerate}
                    \item 學生以校園帳號登入線上書城,系統自動載入其借閱額度、歷史紀錄與閱讀偏好。
                    \item 在搜尋欄輸入關鍵字或課程名稱,系統即時篩選電子書並標示「可借閱」或「暫無授權」狀態。
                    \item 點選電子書後,一鍵申請借閱,系統依出版社授權自動核准,並立即將該書加入學生個人書架。
                    \item 核准後跳出借閱期限與閱讀須知。
                    \item 到期前,首頁與 Email 會以橫幅及推播提醒是否延長借閱或歸還。
                \end{enumerate}
        \end{enumerate}
        
    \item 線上看書,隨時隨地無情讀書電爛同學
        \begin{enumerate}
            \item 情境描述:張小華在通勤途中打開校園線上書城,點選「我的書架」裡剛借閱的《現代資料庫系統概論》電子書,一鍵進入閱讀模式。由於車廂顛簸,他將字體調大並開啟夜間模式,並在第二章重點段落做了螢光筆標註與新增頁邊筆記。下車後,他在校園 Wi-Fi 環境接續閱讀,閱讀進度自動同步到手機與平板,無縫串聯整個學習流程。
            \item 系統應用:
                \begin{enumerate}
                    \item 學生以校園帳號登入後,系統自動載入當前借閱電子書清單與未讀進度。
                    \item 點選「線上閱讀」按鈕,系統開啟可跨裝置的閱讀介面,並根據使用裝置自動調整版面。
                    \item 閱讀工具列支援字體大小調整、頁面亮度切換(夜間/日間模式)及全文搜尋,提升閱讀舒適度。
                    \item 學生在頁面上做螢光標註或新增筆記時,系統即時將標註位置與內容同步至雲端個人筆記區。
                \end{enumerate}
        \end{enumerate}
         
    \item 逾期還書,看書再也不會忘記還書時間
        \begin{enumerate}
            \item 情境描述:張小華因為期末準備繁忙,一整週都在實驗室做報告,沒注意到《現代資料庫系統概論》的借閱期限到了。系統在到期當日凌晨自動幫他完成歸還,並且沒有產生任何罰金或違約記錄。醒來後,他在手機通知欄看到「書籍已自動歸還」的提醒,並能馬上借下一本想讀的電子書,完全不用擔心逾期問題。
            \item 系統應用:
                \begin{enumerate}
                    \item 系統自動計算每本電子書的借閱到期日,並將自動歸還任務排程於到期當日午夜執行。
                    \item 在到期前三天與當日早晨,首頁橫幅與 Email 會提醒學生剩餘借閱時間,讓他有機會提前延長。
                    \item 到期午夜,系統執行歸還流程──更新書籍狀態、釋放借閱額度,並移除個人書架中的該書。
                    \item 完成自動歸還後,系統即時推播與 Email 通知學生「已成功歸還」。
                    \item 系統根據歸還行為更新學生偏好,並在下次登入或推薦中調整借閱提醒頻率,確保更貼心的借閱服務。
                \end{enumerate}
        \end{enumerate}
\end{itemize}