\chapter{資料概念層模型}

\begin{itemize}
    \item 某東西的資料表
        \begin{table}[h!]
            \captionsetup{justification=centering}
            \caption{某東西的資料表}
            \centering
            \renewcommand{\arraystretch}{1.2}
            \begin{tabular}{|c|l|c|c|l|}
                \hline
                \textbf{欄位名稱} & \textbf{說明} & \textbf{資料型態} & \textbf{是否為空} & \textbf{值域} \\
                \hline
                欄位名稱 & 說明 & VARCHAR & 否 & OWASP Top 10 \\
                \hline
                欄位名稱 & 說明 & VARCHAR & 否 & OWASP Top 10 \\
                \hline
                欄位名稱 & 說明 & VARCHAR & 否 & OWASP Top 10 \\
                \hline
                欄位名稱 & 說明 & VARCHAR & 否 & OWASP Top 10 \\
                \hline
                欄位名稱 & 說明 & VARCHAR & 否 & OWASP Top 10 \\
                \hline
                欄位名稱 & 說明 & VARCHAR & 否 & OWASP Top 10 \\
                \hline
                欄位名稱 & 說明 & VARCHAR & 否 & OWASP Top 10 \\
                \hline
                欄位名稱 & 說明 & VARCHAR & 否 & OWASP Top 10 \\
                \hline
                欄位名稱 & 說明 & VARCHAR & 否 & OWASP Top 10 \\
                \hline
                A10:2021 & 說明 & VARCHAR & 否 & OWASP Top 10 \\
                \hline
            \end{tabular}
        \end{table}
\end{itemize}
