\chapter{完整性限制}

\section{學生資料表}

\begin{table}[h!]
\captionsetup{justification=centering}
\caption{學生資料表欄位定義}
\centering
\renewcommand{\arraystretch}{1.2}
\resizebox{\textwidth}{!}{%
    \begin{tabular}{|l|l|l|l|c|}
    \hline
    欄位名稱 & 欄位說明 & 資料型態 & 值域 & 是否為空 \\
    \hline
    UserId & 學生ID & \texttt{VARCHAR(8)} (8 bytes)  & 八位數字或一字母搭配五位數字  & 否 \\
    \hline
    Name   & 姓名     & \texttt{VARCHAR(20)} (20 bytes) & 無特殊符號之非空字串          & 否 \\
    \hline
    Email  & 電子郵件 & \texttt{VARCHAR(255)} (255 bytes)& 符合電子郵件格式的字串        & 否 \\
    \hline
    \end{tabular}%
}
\end{table}
    
\section{書城管理員資料表}

\begin{table}[h!]
    \captionsetup{justification=centering}
    \caption{書城管理員資料表欄位定義}
    \centering
    \renewcommand{\arraystretch}{1.2}
    \resizebox{\textwidth}{!}{%
        \begin{tabular}{|l|l|l|l|c|}
        \hline
        欄位名稱 & 欄位說明 & 資料型態 & 值域 & 是否為空 \\
        \hline
        AdminId & 管理員ID   & \texttt{INT(11) UNSIGNED} (4 bytes) & 僅限正整數           & 否 \\
        \hline
        Name    & 姓名       & \texttt{VARCHAR(20)} (20 bytes)    & 無特殊符號之非空字串 & 否 \\
        \hline
        Email   & 電子郵件   & \texttt{VARCHAR(255)} (255 bytes)  & 符合電子郵件格式的字串 & 否 \\
        \hline
        \end{tabular}%
    }
\end{table}

\section{書籍資料表}

\begin{table}[h!]
    \captionsetup{justification=centering}
    \caption{書籍資料表欄位定義}
    \centering
    \renewcommand{\arraystretch}{1.2}
    \resizebox{\textwidth}{!}{%
        \begin{tabular}{|l|l|l|l|c|}
        \hline
        欄位名稱    & 欄位說明 & 資料型態                      & 值域                       & 是否為空 \\
        \hline
        BookId      & 書籍ID   & \texttt{INT(11) UNSIGNED} (4 bytes)  & 僅限正整數               & 否       \\
        \hline
        Category    & 書籍類型 & \texttt{VARCHAR(255)} (255 bytes)    & 任意非空字串             & 否       \\
        \hline
        Hash        & 雜湊值   & \texttt{VARCHAR(16)} (16 bytes)      & 十六進制字元,英文使用小寫 & 否       \\
        \hline
        ISBN        & 書籍ISBN & \texttt{VARCHAR(13)} (13 bytes)      & 必須符合 ISBN-13 格式    & 否       \\
        \hline
        Amount      & 正本數量 & \texttt{INT(11) UNSIGNED} (4 bytes)  & 僅限正整數               & 否       \\
        \hline
        Publisher   & 出版社   & \texttt{VARCHAR(255)} (255 bytes)    & 任意非空字串             & 否       \\
        \hline
        ReleaseDate & 出版日期 & \texttt{DATE} (3 bytes)               & YYYY-MM-DD               & 否       \\
        \hline
    \end{tabular}%
    }
\end{table}

\clearpage

\section{標題資料表}

\begin{table}[h!]
    \captionsetup{justification=centering}
    \caption{標題資料表欄位定義}
    \centering
    \renewcommand{\arraystretch}{1.2}
    \resizebox{\textwidth}{!}{%
        \begin{tabular}{|l|p{3cm}|l|p{4cm}|c|}
        \hline
        欄位名稱   & 欄位說明                                & 資料型態                              & 值域                    & 是否為空 \\
        \hline
        TitleId    & 標題ID                                 & \texttt{INT(11) UNSIGNED} (4 bytes)   & 僅限正整數              & 否       \\
        \hline
        BookId     & 參照書籍ID                             & \texttt{INT(11) UNSIGNED} (4 bytes)   & 參考 Book 的 BookId     & 否       \\
        \hline
        Language   & ISO 639-1 或 ISO 639-2 格式標記標題語言 & \texttt{VARCHAR(3)} (3 bytes)         & 2~3 個小寫英文字母     & 否       \\
        \hline
        TitleName  & 標題文字                               & \texttt{VARCHAR(255)} (255 bytes)     & 任意非空字串            & 否       \\
        \hline
    \end{tabular}%
    }
\end{table}

\section{借閱副本}

\begin{table}[h!]
    \captionsetup{justification=centering}
    \caption{借閱副本資料表欄位定義}
    \centering
    \renewcommand{\arraystretch}{1.2}
    \resizebox{\textwidth}{!}{%
        \begin{tabular}{|l|p{3cm}|l|p{4cm}|c|}
        \hline
        欄位名稱   & 欄位說明    & 資料型態                          & 值域                       & 是否為空 \\
        \hline
        CopyId     & 副本ID      & \texttt{INT(11) UNSIGNED} (4 bytes)  & 僅限正整數                & 否       \\
        \hline
        Hash       & 雜湊值      & \texttt{VARCHAR(16)} (16 bytes)      & 十六進制字元,英文使用小寫  & 否       \\
        \hline
        OpenDate   & 借閱日期    & \texttt{DATE} (3 bytes)               & YYYY-MM-DD                & 否       \\
        \hline
        ExpireDate & 逾期日期    & \texttt{DATE} (3 bytes)               & YYYY-MM-DD                & 否       \\
        \hline
        UserId     & 擁有者      & \texttt{VARCHAR(8)} (8 bytes)         & 參照 User 的 UserId       & 否       \\
        \hline
    \end{tabular}%
    }
\end{table}

